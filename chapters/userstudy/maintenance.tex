\section{Results: Impact of Verification on Maintenance}
\label{sec:maintain}

Our findings contrast with the conventional
wisdom that verification renders the
code difficult to change and maintain,
as each change requires updating the
\emph{specification} and \emph{proof}
in addition to the code.
%
In particular, our study shows
participants found that formal
specifications enabled the
precise analysis of the \emph{impacts}
of code changes, and hence,
formal verification allowed for
aggressive code \emph{optimizations}
without fear of breaking functionality.

\subsection*{Specification Enables Impact Analysis}

Change impact analysis~\cite{arnold1996introduction} is about determining the implications of code changes. Several participants mentioned how specifications were
a way to scope the effects of change, or as P5 put it,
``another approximation of what will break [when code is changed].''
%
Indeed, P9 explains that this is one of the reasons
they chose to verify their (reference) model in the
first place as it allows them to ``see what happens if
something has changed.''
%
The rationale is that if after a change in the code
the specification fails to hold, participants could
quickly infer what the behavior has been affected.
%
This allowed them to narrow the scope and move from
``changing a whole thing into changing small things,''
as P5 explained, potentially focusing testing efforts
solely on the affected parts.
%
Additionally, repairing or retrofitting the specification
and the proof with ``small fixes'' gave more confidence
to P10 that their change did not have any outstanding effects.

Finally, the specification itself can
be used to \emph{argue for} a change.
%
Participant P7 reported an example where
the specification exposed a ``non-local control flow''
that was not obvious in the code.
%
This resulted in pushing for
a simplification of the code,
which translated to a simpler
specification and proof.
%
Importantly, as P5 pointed out,
all of this is only possible
``after the first [verified] deployment,''
as you can make these changes
by using the existing specification.

\subsection*{Verification Enables Fearless Optimization}

Several participants reported that
verification allowed them to implement,
as P11 put it, ``optimizations that
normal developer[s] would never [consider]''.
%
In fact, P12 explained that it is
``where Dafny really pulled its weight''
as ``localize[d] changes are easy and straightforward,"
meaning that changes that do not deeply affect
the specification can be quickly
and confidently implemented.

\mypara{Optimizing Computations.}
%
P5 described these changes as ``algorithmic optimization'':
they do not modify the end result but instead simplify the underlying
computation.
%
P7 reported using this as way to show results of verification early.
%
They individually verified one of the key functions
of their system, that they then optimized before
verifying the full system.
%
This allowed them to demonstrate the value
of verification and get buy-in from various stakeholders.

\mypara{Optimizing Redundant Checks.}
%
P12 and P13 reported using verification
to remove redundant checks on function
arguments, thereby speeding up the code,
especially if the function is called often.
%
P12 explained that this is made possible
by using \T{requires} clauses, which can
check statically that the arguments are
valid, and thus do not need to check them
at runtime.

\mypara{Modularity and Change.}
%
We observed a tension 
 between breaking Dafny code into smaller pieces
(methods and functions) to make checking faster and avoid
brittle proofs \cref{sec:harden} versus maintainability.
%
For example, P10 noted ``you want methods to be very short […]
So this creates a bunch of work. You have to write a spec for
the intermediate function,'' which can complicate refactoring.
%
It can be difficult to build compact interfaces
when each function is very small, as compact
interfaces might require bundling together multiple functions
---something often avoided to keep proofs small.