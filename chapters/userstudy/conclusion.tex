\section{Conclusion}

We identified key ways that auto-active verification affects software
development processes. In addition to the usual software development steps,
developers must also debug their proofs and harden their code and proofs to keep
them valid in light of future changes. They continue to use traditional
testing approaches, in part because they need to ensure the specifications are
correct and in part because some properties may be left unspecified. However,
verification opens new opportunities for software engineers, who can improve
performance and make other changes with lower risk of introducing regressions.
Developers from formal backgrounds have different expectations of
verification tools and techniques for using them than developers from
software engineering backgrounds. Our findings offer new opportunities to
improve the process of verified software engineering.