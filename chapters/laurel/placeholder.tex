\section{Localizing Missing Assertions}\label{sec:placeholder}

To help the LLM localize the assertion,
\tool finds a potential location for the missing assertion in a failing proof.
%
As a running example throughout this section, we use the lemma in \autoref{fig:union-size},
where both post-conditions are violated,
unless an assertion is added inside the \T{if} branch in lines 7--10.

\begin{figure}
\begin{dafny}
lemma LemmaUnionSize<X>(xs: set<X>, ys: set<X>)
    ensures |xs + ys| >= |xs|
    ensures |xs + ys| >= |ys|
{
    if ys == {} {
    } else {
        var y :| y in ys;
        if y in xs {
            var xr := xs - {y};
            var yr := ys - {y};
            LemmaUnionSize(xr, yr);
        } else {...}
}}
\end{dafny}
\begin{dafnyNoLines}[basicstyle=\footnotesize\ttfamily,xleftmargin=0pt,xrightmargin=0pt]
(*@Error: a postcondition could not be proved on this return path@*)
    if y in xs {
(*@Related location: this is the postcondition that could not be proved@*)
    ensures |xs + ys| >= |xs|
(*@Error: a postcondition could not be proved on this return path@*)
    if y in xs {
(*@Related location: this is the postcondition that could not be proved@*)
    ensures |xs + ys| >= |ys|
\end{dafnyNoLines}
\caption{A failing Dafny lemma with complex control flow.}\label{fig:union-size}
\end{figure}

\begin{figure}
\includegraphics[width=.6\columnwidth]{chapters/laurel/fig/cfg.pdf}
\caption{Control flow graph (CFG) of the lemma in \autoref{fig:union-size}.
The highlighted \T{assert} corresponds to the first violation from the error message.
Orange circles denote candidate placeholder locations, \emph{were the faulty branch absent from the error message}.}\label{fig:cfg}
\end{figure}

At a high level, the localization proceeds in three steps:
\begin{enumerate*}[label=(\arabic*)]
\item \emph{desugaring} the Dafny program into an intermediate representation;
\item \emph{analyzing the error message} to identify the violations and the faulty branch; and
\item \emph{determining the placeholder locations} based on the control flow of the lemma and the information extracted from the error message.
\end{enumerate*}

\paragraph{Intermediate Representation}

To simplify the analysis, \tool operates at the level of intermediate representation
that Dafny uses to generate verification conditions~\cite{LEINO2005209,leino2008boogie}.
%
In this representation, the body of the lemma is represented as an acyclic \emph{control flow graph} (CFG),
and all specifications, calls, and loops are desugared into primitive \T{assume} and \T{assert} statements.
%
For example, the lemma in \autoref{fig:union-size} is desugared into the CFG shown in \autoref{fig:cfg};
here, the \T{assert} statements in node~7 of the CFG correspond to the post-conditions of the original lemma,
and instead of the call in line~10, node~4 simply assumes its post-conditions
(if the call had a pre-condition, it would first be asserted).

\paragraph{Error Message Analysis}

We formalize a Dafny error message as a sequence of \emph{violations},
where each violation is associated with an \T{assert} statement in the CFG that could not be proved.
%
In our example, the error message in \autoref{fig:union-size} contains two violations,
each associated with one \T{assert} statements in basic block~7 of the CFG.
%
\tool handles each violation separately, in the order they appear in the CFG;
for the rest of this section let us focus on the first violation.

For some violations,
the error message also identifies the \emph{faulty branch},
which corresponds to a node in the CFG on the path from the entry to the violation--node 4 in our example.

\paragraph{Placeholder Locations}

Given a violation, \tool determines the set of candidate locations for the assertion placeholder,
based on the following considerations:
\begin{enumerate}[left=0pt]
\item the placeholder should be \emph{as close as possible before} the violation in the CFG;
this allows the assertion to have the most influence on the solver when it tries to prove the violated assertion;
\item if the enclosing CFG node of the violation has \emph{ancestors that contain additional assumptions or local variables},
locations inside those ancestors should also be included,
because assertions placed inside those nodes have access to the additional assumptions and variables;
\item in the presence of the \emph{faulty branch} node,
only locations inside that node should be included.
\end{enumerate}

Returning to our example, if we assume that the error message does not identify the faulty branch,
candidate placeholder locations are denoted with orange circles in \autoref{fig:cfg}.
%
They are located in nodes~2, 4, and 5 because those are the ancestors of node~7 with additional assumptions and variables;
the candidate locations are at the end of their respective nodes,
since this makes them closer to the violation and also ensures that they can benefit from all the assumptions and variables in the node.
%
Once \tool takes into account the faulty branch information, however,
the only remaining candidate location is at the end of node~4.
%
Indeed, adding the assertion \T{assert xr + yr == xs + ys - {y};} at this location fixes the violation;
note that this assertion uses the local variables \T{xr}, \T{yr}, and \T{y} that are in scope at this location,
but not, \eg, inside node~7.
