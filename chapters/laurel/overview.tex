
\section{Overview}\label{sec:overview}


In this section we illustrate the process of generating assertions in \tool,
using a simplified lemma from the \cedar codebase~\cite{Cedar} as a motivating example.
%
\begin{figure}[ht]
\begin{dafny}
lemma ParseDigitsAndDot(s1: string, s2: string)
  requires forall i | 0 <= i < |s1| :: '0' <= s1[] <= '9'
  ensures ParseDecStr(s1+"."+s2).value.0 == s1 && ParseDecStr(s1+"."+s2).value.1 == "."+s2
  {
    if |s1| == 0 {
    } else {
      ParseDigitsAndDot(s1[1..],s2);
    }
  }
\end{dafny}
\begin{dafnyNoLines}[basicstyle=\footnotesize\ttfamily,xleftmargin=0pt,xrightmargin=0pt]
(*@Error: a postcondition could not be proved on this return path@*)
      } else {

(*@Related location: this is the postcondition that could not be proved@*)
ensures ParseDecStr(s1+"."+s2).value.0 == s1 &&
        ^^^^^^^^^^^^^^^^^^^^^^^^^^^^^^^^^^^^
        ParseDecStr(s1+"."+s2).value.1 == "."+s2
\end{dafnyNoLines}
\caption{Example of a Dafny lemma that fails to verify without an assertion and the corresponding error message.}\label{fig:error}
\end{figure}

Consider the lemma \T{ParseDigitsAndDot} in \autoref{fig:error},
which specifies a correctness property of the function \T{ParseDecStr}:
namely, given an input string of the form \T{s1+"."+s2},
where \T{s1} contains only digits,
the function should return a pair \T{(s1, "."+s2)}
(\eg \T{ParseDecStr("123.456")} should return \T{("123", ".456")}).%
\footnote{The actual \cedar lemma requires that \finlinedafny{s1} be non-empty,
but we elide this requirement for exposition purposes.}

The programmer decides to prove this lemma by induction over the length of \T{s1};
to this end, they introduce a conditional to separate the base case from the inductive case,
and in the latter case, make a recursive call on the tail of \T{s1} to invoke the induction hypothesis.
%
Unfortunately, this is not enough to convince the Dafny verifier,
which returns the error message shown at the bottom of the figure.
%
According to this error message,
the solver is unable to prove that (the first conjunct of) the post-condition in Line~2
is established by the \T{else} branch (\ie the recursive case).
%
The reason for this error is not that the lemma is incorrect,
but rather that the chain of reasoning steps required to deduce the postcondition from the induction hypothesis is too long,
and the solver gets stuck along the way.
%
We can help the solver by providing an intermediate ``checkpoint'' that helps break down the proof;
more precisely, we need to insert the following \emph{assertion} at the end of the \T{else} branch (after Line~6):
\begin{equation}\label{eq:helper}
\mbox{\T{assert s1+"."+s2 == [s1[0]]+(s1[1..]+"."+s2)}}
\end{equation}
%
This assertion explains the relationship between the string passed to \T{ParseDecStr} in the current context
and its version passed to the recursive call.
%
It is challenging for the programmer to come up with this assertion
(unless they have extensive experience with similar proofs);
specifically, the error message provides very little insight into why the solver got stuck.


\paragraph{Can LLMs Generate Assertions?}

Encouraged by the success of LLMs in program repair~\cite{khatry2023words,ProgramRepairXia2023}
and proof synthesis~\cite{BaldurFSE23First,DafnyFSE24,kamath2023Invariant},
we conjecture that LLMs can be used to generate assertions automatically.
%
Unfortunately, na\"ively prompting an LLM with the code snippet and error message in \autoref{fig:error},
and asking it to fix this code by inserting an assertion does not work well.
%
Here are three sample assertions this \naive baseline yields in our experiments:
\begin{enumerate}[left=0pt]
  \item \T{assert forall i | 0 <= i < |s1| :: '0' <= s1[i] <= '9'} after Line~3;
  \item \T{assert '0' <= s1[0] <= '9'} after Line~4;
  \item \T{assert s2 != ""} after Line~5.
\end{enumerate}
%
Neither of the three assertions is helpful in proving the lemma
(and the last assertion also does not hold),
and moreover, they are not even inserted at the correct location!
%
We find this last result quite surprising,
since the placement of assertions---unlike their content---is usually easy to determine for human programmers
based on the error message.
%

\paragraph{The \tool Approach}

\begin{figure*}
  \centering
  \includegraphics[width=1\textwidth]{chapters/laurel/fig/laurel_diagram.pdf}
  \caption{Overview of \tool: it takes two inputs,
  the lemma to fix, and the codebase used to find \emph{in-context examples}.
  It identifies the assertion location in the lemma’s \T{else} branch using
  the error message and selects relevant examples with a similar recursive structure.
  Finally, it prompts the LLM to generate an assertion.}
  \label{fig:overview}
\end{figure*}

To improve the success rate of LLMs in generating assertions,
we propose to augment the prompt with two crucial pieces of information:
\begin{enumerate*}
  \item the most likely \emph{location} where the assertion should be inserted;
  \item a set of relevant \emph{example assertions} from the same codebase.
\end{enumerate*}
%
We implement this approach in a tool called \tool,
whose high-level overview is shown in \autoref{fig:overview}.

\tool starts with two inputs:
\begin{enumerate*}[label=\alph*)]
  \item a Dafny codebase
  \item the name of the target lemma.%
  \footnote{A verification unit in Dafny is either a \emph{lemma} or a \emph{method}, depending on the presence of executable code;
  we refer to both as ``lemma'' for simplicity.}
\end{enumerate*}
%
The goal is to insert an assertion into the target lemma so that it verifies.
%
In this work, we assume that the lemma can be fixed by inserting a single assertion;
\autoref{sec:eval:threats} discusses how our approach might be extended to handle multiple assertions.

\tool constructs the LLM prompt in two key steps.
%
First, it runs the Dafny verifier on the target lemma to obtain the error message.
%
It then analyzes the error message to determine the most likely location for the assertion;
this location is used to insert an \emph{assertion placeholder} into the lemma,
as shown in \autoref{fig:overview}.
%
Second, \tool analyzes the codebase to identify a small set of $m$ relevant assertions
($m=6$ in our experiments)
to be used as \emph{in-context examples} (also called few-shot demonstrations~\cite{fewshot}).
%
\tool then constructs the prompt, including the lemma with the assertion placeholder and the example assertions,
and asking the LLM to generate an assertion to be inserted at the placeholder location.

Given an output from the LLM,
\tool replaces the placeholder with the generated assertion and re-runs the Dafny verifier;
if the target lemma still fails to verify,
\tool re-queries the LLM with the same prompt up to a fixed number $k$ of attempts
($k=10$ in our experiments).%
\footnote{In our experiments, to save cost, we sample $k$ outputs from the LLM at once, instead of querying the LLM $k$ times.}
%
The rest of this section gives a brief overview of our techniques for placeholder localization and example assertion selection;
\autoref{sec:placeholder} and \autoref{sec:lemma_similarity} provide more details.

\paragraph{Localizing the Assertion Placeholder}

Recall that with a \naive prompt,
the LLM struggles to determine the correct location for the missing assertion in our motivating example.
%
Our \textbf{first key insight} is that we can help the LLM localize the assertion by analyzing the error message.
%
Specifically, the error message in \autoref{fig:error} points to a \emph{violation} at the very end of the lemma
(which is where post-conditions are checked);
it also points to the \emph{faulty branch}---the \T{else} branch---which led to the violation.
%
Given this information, the optimal location for the assertion is at the end of the \T{else} branch:
placing it outside of the conditional would make assertion (\ref{eq:helper}) ill-formed
(since \T{s1} might be empty, so \T{s1[0]} might not exist);
on the other hand, placing it further away from the violation (\eg before the recursive call)
might not be as helpful, as at that point the solver is busy with something else (proving the preconditions of the recursive call).
%
\tool performs this analysis based on the control-flow graph of the lemma,
as we explain in \autoref{sec:placeholder}.

\paragraph{Selecting Example Assertions}

Once we know \emph{where} to insert the assertion,
the next challenge is to decide \emph{what} it should say.
%
Sometimes the missing assertion is syntactically similar to the failing specification
or to other assertions within the same lemma.
%
In these cases, the LLM can often generate the correct assertion
by taking hints from the error message or from the lemma itself.
%
This is not always the case, however,
and it is not the case in our motivating example.


Our \textbf{second key insight} is that the missing assertion is likely similar to other assertions
elsewhere in the same codebase,
which we can make available to the LLM as \emph{in-context examples}.
%
The challenge, however, is to select a small number of examples
that are \emph{relevant} to the current context
(that is, are likely to be syntactically similar to the missing assertion).
%
We further conjecture that relevant example assertions are likely to appear in proofs that are similar to our failing proof.
%
For instance, in our motivating example,
we might consider other lemmas that involve string manipulation
and have the structure of an inductive proof (with a base case and a recursive case).

To find such similar lemmas, \tool introduces a code similarity metric,
based on a hierarchical generalization of Levenshtein distance (\autoref{sec:lemma_similarity}),
which is effective at capturing structural similarity between code snippets.
%
\tool then selects $m$ assertions from the codebase whose context---%
\ie the surrounding lemma with the assertion removed---%
is most similar to our target lemma.

In our motivating example,
this algorithm recommends the following assertion from a lemma with a similar recursive structure:
%
\begin{equation*}
  \resizebox{\textwidth}{!}{\T{assert wrapNumSeg(s)+[DC] == wrapNumSegs([s[0]]) + (wrapNumSegs(s[1..]) + [DC])}}
 \end{equation*}
%
You can see that this assertion is syntactically similar to the missing assertion (\ref{eq:helper}).
%
Given this assertion as an example,
together with the assertion placeholder,
the LLM can now generate the correct assertion for the motivating example.